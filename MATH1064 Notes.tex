\documentclass[12pt]{article}
\NeedsTeXFormat{LaTeX2e}

%%%%%%%%%%%%%%%%%%%%%%%%%%%%%%%%%%%%%%%%%

% Template by Nicholas Bertollo
% You may change and reuse this as you see fit

%%%%%%%%%%%%%%%%%%%%%%%%%%%%%%%%%%%%%%%%%

% Replace this information!
\newcommand{\studyunit}{MATH1064}
\newcommand{\studyperiod}{Semester 2, 2025}

%%%%%%%%%%%%%%%%%%%%%%%%%%%%%%%%%%%%%%%%%

\usepackage[svgnames]{xcolor}
\usepackage[T1]{fontenc}
\usepackage[margin=2.7cm,a4paper]{geometry}

\usepackage[osf]{mathpazo}
\usepackage{amsmath,amsthm,amsfonts,amssymb,mathtools}

\usepackage{hyperref,url}
\usepackage{xstring}

\usepackage{environ}
\usepackage{tasks}

\usepackage{etoolbox}
\usepackage{fourier-orns}
\usepackage{kvoptions}
\usepackage[]{units}
\usepackage[normal]{subfigure}

% Algorithms package

% \usepackage[ruled]{algorithm2e} % You can use the algorithm2e if you'd like
\usepackage[noend]{algpseudocode} % You may get rid of noend if you like. 
\usepackage{algorithm}
\usepackage{algorithmicx}

% No indents
\usepackage[parfill]{parskip}

% Tree diagram
\usepackage{tikz}
\usetikzlibrary{shapes.geometric}

% Graphics and images
\usepackage{graphicx}
\graphicspath{ {./images/} }

% Header
\usepackage{fancyhdr}
\addtolength{\headheight}{2.5pt}
\pagestyle{fancy}
\fancyhead{} 
\fancyhead[L]{\sc \studyunit}
\fancyhead[R]{\sc \studyperiod}
\fancyfoot{}
\fancyfoot[R]{\thepage}
\renewcommand{\headrulewidth}{0.75pt}

%%%%%%%%%%%%%%%%%%%%%%%%%%%%%%%%%%%%%%%%%%%%%%%%%%%%%%%%%%%%%%%%%%%%%%%%%%%%%%%%%%

% Some basic mathematics commands!

\DeclarePairedDelimiter\ceil{\lceil}{\rceil}
\DeclarePairedDelimiter\abs{\lvert}{\rvert}

%%%%%%%%%%%%%%%%%%%%%%%%%%%%%%%%%%%%%%%%%%%%%%%%%%%%%%%%%%%%%%%%%%%%%%%%%%%%%%%%%%

% Macros
\newcounter{weekcntr}
\newcommand*{\week}{
    \stepcounter{weekcntr}
    \section*{Week \theweekcntr}
}
    
 
\begin{document}
    \begin{titlepage}
        \begin{center}
            \vspace*{1cm}

            \section*{MATH1064 Study Notes}

            \vspace{0.25cm}
            University of Sydney \\
            \studyperiod \\
            \studyunit

            \vfill
            \includegraphics{thumbs_up_emoji.png}
            \vspace{1.5cm}

            Happy studying!
            
       \vspace{0.8cm}
            
        \end{center}
    \end{titlepage}
    
    \week{}
    \subsubsection*{Introduction to Discrete Maths}

        Discrete maths is the study of "discrete structures", this includes objects which are:
        \begin{itemize}
            \item Countable or listable
            \item Distinct and unconnected.
        \end{itemize}

        There are two objectives of {\studyunit}:
        \begin{enumerate}
            \item Develop mathematical reasoning skills. This involves using \textbf{logic and proofs}, and 
            \textbf{rigorously} (exhausting all possibilities) finding solutions to problems.
            \item Study \textbf{discrete structures} and their properties including:
            \begin{itemize}
                \item Sets and functions
                \item Prime numbers and modular arithmetic
                \item Graphs and networks
                \item Counting and probability.
            \end{itemize}
        \end{enumerate}

    \newpage
    \week{}

    \newpage
    \week{}

    \newpage
    \week{}

    \newpage
    \week{}
    
\end{document}