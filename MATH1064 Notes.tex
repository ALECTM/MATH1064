\documentclass[12pt]{article}
\NeedsTeXFormat{LaTeX2e}

%%%%%%%%%%%%%%%%%%%%%%%%%%%%%%%%%%%%%%%%%

% Template by Nicholas Bertollo
% You may change and reuse this as you see fit

%%%%%%%%%%%%%%%%%%%%%%%%%%%%%%%%%%%%%%%%%

% Replace this information!
\newcommand{\studyunit}{MATH1064}
\newcommand{\studyperiod}{Semester 2, 2025}

%%%%%%%%%%%%%%%%%%%%%%%%%%%%%%%%%%%%%%%%%

\usepackage[svgnames]{xcolor}
\usepackage[T1]{fontenc}
\usepackage[margin=2.7cm,a4paper]{geometry}
\renewcommand{\baselinestretch}{1.15}

\usepackage[osf]{mathpazo}
\usepackage{amsmath,amsthm,amsfonts,amssymb,mathtools}

\usepackage{hyperref,url}
\usepackage{xstring}

\usepackage{environ}
\usepackage{tasks}

\usepackage{etoolbox}
\usepackage{fourier-orns}
\usepackage{kvoptions}
\usepackage[]{units}
\usepackage[normal]{subfigure}

% Algorithms package

% \usepackage[ruled]{algorithm2e} % You can use the algorithm2e if you'd like
\usepackage[noend]{algpseudocode} % You may get rid of noend if you like. 
\usepackage{algorithm}
\usepackage{algorithmicx}

% No indents
\usepackage[parfill]{parskip}

% Tree diagram
\usepackage{tikz}
\usetikzlibrary{shapes.geometric}

% Graphics and images
\usepackage{graphicx}
\graphicspath{ {./images/} }

% Header
\usepackage{fancyhdr}
\addtolength{\headheight}{2.5pt}
\pagestyle{fancy}
\fancyhead{} 
\fancyhead[L]{\sc \studyunit}
\fancyhead[R]{\sc \studyperiod}
\fancyfoot{}
\fancyfoot[R]{\thepage}
\renewcommand{\headrulewidth}{0.75pt}

%%%%%%%%%%%%%%%%%%%%%%%%%%%%%%%%%%%%%%%%%%%%%%%%%%%%%%%%%%%%%%%%%%%%%%%%%%%%%%%%%%

% Some basic mathematics commands!

\DeclarePairedDelimiter\ceil{\lceil}{\rceil}
\DeclarePairedDelimiter\abs{\lvert}{\rvert}

\newcommand{\Z}{\mathbb{Z}}
\newcommand{\N}{\mathbb{N}}
\newcommand{\R}{\mathbb{R}}
\newtheorem{theorem}{Theorem}
\theoremstyle{definition}
\newtheorem{definition}{Definition}

%%%%%%%%%%%%%%%%%%%%%%%%%%%%%%%%%%%%%%%%%%%%%%%%%%%%%%%%%%%%%%%%%%%%%%%%%%%%%%%%%%

% Macros
% \newcounter{weekcntr}
% \newcommand*{\week}{
%     \stepcounter{weekcntr}
%     \section{Week \theweekcntr}
% }
    
 
\begin{document}
    \begin{titlepage}
        \begin{center}
            \vspace*{1cm}

            \section*{MATH1064 Study Notes}

            \vspace{0.25cm}
            University of Sydney \\
            \studyperiod \\
            \studyunit

            \vfill
            \includegraphics{thumbs_up_emoji.png}
            \vspace{1.5cm}

            Happy studying!
            
       \vspace{0.8cm}
            
        \end{center}
    \end{titlepage}

    \tableofcontents
    
    \newpage
    \section{Introduction to Discrete Maths and Set Theory}
    \subsection{Introduction to Discrete Maths}

        Discrete maths is the study of "discrete structures", this includes objects which are:
        \begin{itemize}
            \item Countable or listable
            \item Distinct and unconnected.
        \end{itemize}

        There are two objectives of {\studyunit}:
        \begin{enumerate}
            \item Develop mathematical reasoning skills. This involves using \textbf{logic and proofs}, and 
            \textbf{rigorously} (exhausting all possibilities) finding solutions to problems.
            \item Study \textbf{discrete structures} and their properties including:
            \begin{itemize}
                \item Sets and functions
                \item Prime numbers and modular arithmetic
                \item Graphs and networks
                \item Counting and probability.
            \end{itemize}
        \end{enumerate}
    
        \subsection{Introduction to Set Theory}
        \subsubsection{Definitions}
        \begin{definition}[Set]
            \label{def:set}
            A \textbf{set} $S$ is a collection of objects, called \textbf{elements} of $S$.
        \end{definition}

        \begin{itemize}
                \item If $x$ is an element of $S$, $x \in S$
                \item If not, then $x \notin S$.
            \end{itemize}

        Sets can be finite or infinite:
        \begin{itemize}
            \item Example (finite): $S = \{2, 3, 5\}$
            \item Example (infinite): $S = \{0, 1, 2, \dots\} = \N$ \\
        \end{itemize}

        \begin{definition}[Set Equivalence]
            Two sets $S$ and $T$ are said to be equal if they contain the same elements, regardless of \textbf{order} or 
            \textbf{repetition}.
        \end{definition}

        \begin{itemize}
            \item Example 1: \\
            \begin{align*}
                S &= \{1,1,5,3\} \\
                T &= \{1,3,5\} \\
                S &= T
            \end{align*}
            \item Example 2:
            \begin{align*}
                S &= \{-1, 0, 1, \dots\} \\
                T &= \{0, 1, \dots\} \\
                S &\neq T
            \end{align*}
        \end{itemize}

        \begin{definition}[Empty Set]
            \label{def:empty-set}
            The \textbf{empty set} is a unique set containing no elements. $\emptyset = \{\}$.\\
        \end{definition}
        \begin{definition}[Singlton Set]
            \label{def:singleton-set}
            A \textbf{singleton set} has only one element, e.g. $S = \{x\}$ or $S = \{x, x, x\}$
        \end{definition}

        \subsubsection{Unpacking Sets}
        Sets can contain other sets as elements, inner sets are considered distinct elements even if 
        their contents are the same as other elements in the outer set. This is because when we unpack sets,
        we only remove outer curly braces \{ \}.
        \begin{align*}
            S &= \{1, 2, \{1\} \} \\
            \abs{S} &= 3 \text{ distinct elements}
        \end{align*}

        \subsubsection{Sets with Properties}
        We can describe sets using \textbf{set builder notion} which indicates the properties of a set.
        \begin{align*}
            A = \{x \in S \mid P(x) \}
        \end{align*}
        "The set A consists of all elements $x$ of $S$ such that $x$ has property $P$".\\\\
        Examples:
            \begin{align*}
                \{x \in \N \mid 3 \le x \le 5\} &= \{3,4,5\} \\
                \{y \in \Z \mid y=2k \text{ for some } k \in \Z\} &= \{\dots, -2, 0, 2, \dots\} \\
                \{2z + 1 \mid z \in \N\} &= \{1, 3, 5, \dots\}
            \end{align*}
        
            \subsubsection{Russel's Paradox}
            Define a set $T = \{S, set \mid S \notin S\}$. The set $T$ contains any set $S$ which does not contain itself.
            
            Consider if $T \in T$:
            \begin{itemize}
                \item If $T \in T$: $T$ does not satisfy the condition.
                \item If $T \notin T$: $T$ does satisfy the condition, thus $T \in T$ according to our definition.
            \end{itemize}
            This induces a contradiction, hence demonstrating that we need \textbf{axioms} which \textbf{rigorously} state
            how to define and build sets.
        
            \subsubsection{Operations on Sets}
            \begin{definition}[Union]
                \label{def:union}
                Given two sets $S$ and $T$, the \textbf{union} of $S$ and $T$ is the set containing all elements from $S$ and $T$.
                This is written as $S \cup T$ where $x \in S$ \textbf{OR} $x \in T$.
            \end{definition}
            \begin{itemize}
                \item Example 1: $\{1,2,3\} \cup \{2,5\} = \{1,2,3,5\}$
                \item Example 2: $\{0,1,2,\dots\} \cup \{0,-1,-2,\dots\} = \Z$ \\
            \end{itemize}

            \begin{definition}[Intersection]
                \label{def:intersection}
                Given two sets $S$ and $T$, the \textbf{intersection} of $S$ and $T$ is the set of elements belonging to
                both $S$ and $T$. This is written as $S \cap T$ where $x \in S$ \textbf{AND} $x \in T$.
            \end{definition}
            \begin{itemize}
                \item Example 1: $\{1,2,3\} \cap \{2,5\} = \{2\}$
                \item Example 2: $\{1,2,3,\dots\} \cap \{-1,-2,-3,\dots\} = \emptyset$ \\
            \end{itemize}

            Multiple unions and intersections can be taken at a time.
            \begin{align*}
                \displaystyle\bigcup_{i=1}^{3} \{i, 2i\} &= \{1,2\} \cup \{2,4\} \cup \{3,6\} \\
                &= \{1,2,3,4,6\}
            \end{align*}
            \begin{align*}
                \displaystyle\bigcap_{i=1}^{3} \{i, i+1, i+2\} &= \{1,2,3\} \cap \{2,3,4\} \cap \{3,4,5\} \\
                &=\{3\}
            \end{align*}
            
            Formally, for sets $A_1,A_2,\dots,A_i$ we define the set $A_i$ as:
            \begin{equation*}
                \displaystyle\bigcup_{i=1}^{\infty} A_i = \{x \mid x \in A_{k} \text{ for SOME } k \ge 1\}
            \end{equation*}
            That is for an infinite series of unions, $x$ is an element that appears \textbf{at least once} 
            in the sets $A_k$ for $k \ge 1$.

            And similarly for intersections, we define the set $A_i$ as:
            \begin{equation*}
                \displaystyle\bigcap_{i=1}^{\infty} A_i = \{x \mid x \in A_{k} \text{ for ALL } k \ge 1\}
            \end{equation*}
            That is, for an infinite series of intersections, $x$ is an element which appears in \textbf{all} 
            sets $A_k$ for $k \ge 1$.

            
            \subsubsection{Subsets}
            \begin{definition}[Subsets]
                \label{def:subset}
                A set $S$ is a \textbf{subset} of another set $T$ if every element of $S$ is an element of $T$. This
                is written as $S \subset T$.
            \end{definition}

            Additionally, if $S$ is a subset of $T$ but is not equal to $T$, then it is considered a \textbf{proper subset},
            denoted as $S \subseteq T$.
            
            For example,
            \begin{align*}
                S &= \{2,4,6\} \\
                T &= \{2,4,6,8\}\\
                S &\subseteq T, \text{ since $8 \notin S$}
            \end{align*}


            \subsubsection{Proving Subset Relationships}
            To prove $S \subseteq T$, we need to:
            \begin{enumerate}
                \item Take an arbitrary element of $S$, which we  call $x$
                \item Show that $x \in T$ \\
            \end{enumerate}

            \textbf{Example:} Let $S$ and $T$ be sets where,
            \begin{align*}
                S &= \{2n \mid n \in \N, n \ge 1\} \\
                T &= \{2^{m} \mid m \in \N\}
            \end{align*}
            \begin{proof}
                Let $x \in S$, by definition $x=2^n$ for some $n \ge 1$.
                \begin{align*}
                    x &=2^n \\
                    x &=2(2^{n-1}), \text{ rewriting $x$}
                \end{align*}
                Since $n \ge 1$, $n-1 \ge 0$ meaning that $n-1 \in \N$ $\implies 2^{n-1} \in \N$.
                Because $2^{n-1}$ is a natural number, we can rewrite $x = 2m$ where $m=2^{n-1}$ as we know
                $m \in \N$, $x \in T \implies S \subseteq T$.
            \end{proof}

            \subsubsection{Proving Equality Relationships}
            To prove that $S=T$, we need a \textbf{"double containment proof"} which shows that both sets have the same
            elements, that is:
            \begin{enumerate}
                \item Every $x \in S$ also satisifes $x \in T$
                \item Every $x \in T$ also satisfies $x \in S$
            \end{enumerate}
            \textbf{Example:} Let $S$ and $T$ be sets where,
            \begin{align*}
                S &= \{2m+1 \mid m \in \Z\} \\
                T &= \{2r-1 \mid r \in \Z\}
            \end{align*}
            \begin{proof}[Proof. Show $S \subseteq T$]
                Let $x \in S$, then $x = 2m+1$ for some $m \in Z$. \\
                Let $r = m+1$.
                \begin{align*}
                    x &= 2m + 1 \\
                    x &= 2m + 2 - 1, \text{ rewriting $x$} \\
                    x &= 2(m + 1) - 1, \text{ notice $m+1 = r$} \\
                    x &= 2r - 1, \text{this is the same as $x \in T$}
                \end{align*}
                $x = 2r -1$ for some $r \in \Z$. Thus, $x \in T \implies S \subseteq T$.
            \end{proof}

            \begin{proof}[Proof. Show $T \subset S$]
                Let $x \in T$, then $x = 2r-1$ for some $r \in \Z$. \\
                Let $m = r - 1$.
                \begin{align*}
                    x &= 2r-1 \\
                    x &= 2r - 2 + 1 \\
                    x &= 2(r-1) + 1 \\
                    x &= 2m + 1
                \end{align*}
                $x = 2m + 1$ for some $m \in \Z$. Thus, $x \in S \implies T \subseteq T$.
            \end{proof}
            Since $S \subseteq T$ and $T \subseteq S$, the two sets must have the same elements and are equal, $S = T$.


            \subsection{More Set Theory}
            \subsubsection{Cardinality}
            The \textbf{cardinality} of a set $S$ in a rough sense refers to the size of $S$, i.e. the no. of elements in
            $S$.
            \begin{itemize}
                \item If $S$ is finite, then $\abs{S}$ is the number of distinct elements in $S$.
                \item If $S$ is infinite, then we write $\abs{S} = \infty$. 
            \end{itemize}
            Note that there can be \textbf{different infinite cardinalities}, or sizes of infinities. A basic example of this
            is the set of natural numbers $\N$ compared to the set of real numbers $\R$.

            \subsubsection{Set Differences}
            \begin{definition}[Set Difference]
                \label{def:set-difference}
                Given two sets $S$ and $T$, the \textbf{set difference} is the set of elements $x \in S$ and $x \notin S$.
                This is written as $S \backslash T$ or $S - T$.
            \end{definition}
            \begin{itemize}
                \item Example 1: $\{1,2,3\}\backslash \{2, 5\} = \{1, 3\}$
                \item Example 2: $\{0, 1, 2, \dots\} \backslash \{0, -1, -2, \dots\} = \{1, 2, \dots\}$
                \item Example 3: $\N \backslash \Z = \emptyset$
            \end{itemize}

    \newpage
    \section{Week 2}

    \newpage
    \section{Week 3}

    \newpage
    \section{Week 4}

    \newpage
    \section{Week 5}
    
\end{document}